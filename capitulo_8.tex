\chapter{Capitulo 8 Ventajas y Desventajas del proceso}
\label{capitulo 3}


\section{ \textbf{Ventajas}: }
El sistema de deposición automática de fibra tiene la capacidad de variar la velocidad de colocación, la presión, la temperatura y la tensión de las fibras. Todas estas capacidades se complementan con un sistema de programación fuera de línea que beneficiaría el tiempo de producción de la máquina. 
\item 1. Plazos de producción más cortos: las pruebas toman tiempo, pero acortan el plazo total de comercialización y permiten a las partes identificar problemas y abordarlos tempranamente. Encontrar estos problemas más adelante a menudo genera retrasos y cambios costosos e incluso podría exigir rediseños completos de los procesos o equipos de producción.

\item2. Tasas de colocación más rápidas: la velocidad a la que se aplican los materiales compuestos es fundamental. Las rápidas tasas de disposición del material producen una mayor fabricación en todo momento. Las primeras pruebas permiten calibrar los procesos de automatización y formateo de materiales para lograr el proceso de colocación más rápido posible.

\item 3. Menos paradas y cambios: el funcionamiento continuo es tan importante como la velocidad. Al incluir un formateo de precisión como parte del proceso de desarrollo de productos, se pueden minimizar las paradas no planificadas, costosas y que consumen mucho tiempo.

Por ejemplo, el formato y el tamaño del carrete son factores clave en los cambios. El diseño del carrete debe basarse en el tamaño de la pieza, el cronograma de producción y las limitaciones de tiempo para la aplicación específica. Si se necesitan 100 pies de cinta cortada compuesta para fabricar una pieza y se producen seis piezas por día, los carretes deben transportar al menos 600 pies de cinta cortada. De lo contrario, será necesario realizar cambios durante un día de producción, lo que provocará retrasos innecesarios. Además, si el tamaño del carrete es incorrecto para una aplicación, puede surgir un problema de gestión de materiales con carretes remanentes o material costoso desperdiciado debido al tiempo de espera. Parece sencillo, pero incluso este tipo fundamental de planificación afecta al fabricante de materiales y al proveedor de equipos de automatización.
La mala calidad de la cinta también puede provocar tiempos de inactividad. Las cintas que no han sido formateadas correctamente pueden tener bordes irregulares y peludos o tiras que pueden quedar atrapadas y apagar la máquina AFP. Resolver los problemas de tamaño de carrete y calidad de la cinta antes de la producción puede evitar paradas y cambios innecesarios.4. Mayor rendimiento del material: el rendimiento del material generalmente depende de las especificaciones del rollo principal frente al ancho y largo de la cinta formateada, el tamaño y la forma de la pieza fabricada y el tipo de empalmes aceptables en la pieza terminada; también implica compensaciones, como la elección entre cinta compuesta ancha o estrecha.
 Una cinta más ancha permite colocar más material con cada pasada de la máquina, pero también podría haber más desperdicio en los bordes con dientes de sierra. Una cinta más estrecha podría mejorar el rendimiento, pero a costa de más tiempo de producción. Crear anchos de hendidura óptimos y una cantidad de cintas por disposición son vitales para cumplir con los objetivos de rendimiento y costos de producción. La combinación de varios tamaños también puede mejorar el rendimiento sin comprometer el rendimiento.
Resolver estos problemas durante el desarrollo de productos conserva los materiales compuestos, mejora la relación compra-vuelo y reduce los costos de producción. La realización de pruebas tempranas permite a los socios de la cadena de suministro identificar y superar los desafíos y aprovechar las oportunidades para utilizar nuevos materiales, formatos y métodos de fabricación.
\section{ \textbf{Desventajas}: }
En este proceso es imposible controlar de manera precisa el espesor del material compuesto ya que incluso si la pieza sobre la cual se posicionan las fibras tiene una forma cilíndrica perfecta el borde interior del camino curvo no será cubierta de manera perfecta, cuando las imperfecciones de el carrete de fibra como la diferencia del borde se agregan las longitudes y la variación del ancho la distorsión del elemento empeora, en principio, los defectos inducidos por la deformación por flexión en el plano, como la deformación local por pandeo y el cambio de espesor son inevitables, por lo tanto se recomienda que la curvatura del camino del carrete se mantenga al mínimo posible para reducir estos efectos locales. La mayoría de los defectos son provocados por flexión en el plano sobre el cuál se aplican las fibras, el espacio entre las fibras y la superposición de estas provocan que la deformación sea inevitable y aunque la uniformidad del espesor puede ser superior a otros métodos sigue sin ser controlado de manera perfecta.

\textbf{}