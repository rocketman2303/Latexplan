\setcounter{page}{1}
\chapter*{Resumen}
\addcontentsline{toc}{chapter}{Resumen}


El concepto de colocación de fibras hace referencia a los procedimientos de manufactura de materiales compuestos que implican la disposición de fibras de refuerzo a lo largo de trayectorias predefinidas en el componente (Bannister, 2001; Crosky et al., 2012). También conocido como colocación dirigida de fibras y dirección de fibras, este proceso consta de dos componentes estrechamente interrelacionados: (i) la definición de ubicaciones y trayectorias del refuerzo de fibra y (ii) la selección del método de fabricación. Las restricciones y limitaciones inherentes al método de fabricación desempeñan un papel crucial en la definición de las trayectorias, si bien la configuración deseada del refuerzo también ejerce una notable influencia en la elección del método de fabricación.

Existen varios procesos automatizados que podrían considerarse como formas de colocación de fibras, tales como el bobinado de filamentos, la pultrusión y el preformado (Peters, 2011; Starr, 2000; Tong et al., 2002); sin embargo, estos no se abordan en este contexto. Este capítulo se enfoca, en cambio, en las tecnologías de colocación de fibras capaces de posicionar haces de fibras o cintas en diversas orientaciones mediante el uso de diversos dispositivos. Si bien la colocación de fibras puede emplearse para la fabricación de piezas completas, también puede llevarse a cabo de manera localizada, como en el refuerzo alrededor de orificios de pernos y recortes (Bannister, 2001).

La obtención de las trayectorias de las fibras se logra mediante diversos enfoques. Se han empleado métodos analíticos, incluyendo los enfoques basados en los esfuerzos principales y en la trayectoria de la carga, así como técnicas de optimización, como el uso de algoritmos genéticos (AG). Estos métodos se detallan a continuación y se ejemplifican mediante un análisis de un agujero sometido a carga por un pasador. Independientemente del método utilizado, es imperativo ajustarse a las restricciones de los procesos de fabricación, tales como el radio mínimo en una trayectoria dada, y considerar la necesidad de incluir capas adicionales en el laminado para prevenir fallos asociados a trayectorias secundarias de carga. Además, se deben contemplar las deformaciones derivadas de las tensiones de curado y los requisitos de servicio, como los daños provocados por granizo y piedras.